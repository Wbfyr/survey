\documentclass[conference]{IEEEtran}
\IEEEoverridecommandlockouts
% The preceding line is only needed to identify funding in the first footnote. If that is unneeded, please comment it out.
\usepackage{ctex}
\usepackage{algorithm}
\usepackage[noend]{algpseudocode}
\usepackage{cite}
\usepackage{amsmath,amssymb,amsfonts}
\usepackage{graphicx}
\usepackage{textcomp}
\usepackage{xcolor}
\def\BibTeX{{\rm B\kern-.05em{\sc i\kern-.025em b}\kern-.08em
    T\kern-.1667em\lower.7ex\hbox{E}\kern-.125emX}}
\begin{document}

\title{Resource Management in Large Language Models\\
    \thanks{Identify applicable funding agency here. If none, delete this.}
}

\author{\IEEEauthorblockN{{\kaishu 朱徐洲}}
    \IEEEauthorblockA{522031910324}
    \IEEEauthorblockA{zhuxuzhou@sjtu.edu.cn}
    \and
    \IEEEauthorblockN{{\kaishu 庞皓阳}}
    \IEEEauthorblockA{522031910050}
    \IEEEauthorblockA{1271362703@sjtu.edu.cn}
    \and
    \IEEEauthorblockN{{\kaishu 卢晗宇}}
    \IEEEauthorblockA{522031910043}
    \IEEEauthorblockA{lhy-2785.fanyang@sjtu.edu.cn}
}

\maketitle

\begin{abstract}
    In recent years, there has been a continuous evolution in the scale, performance, and applications of large language models (LLMs),\\
    progressing from initial models with several million parameters to current models with tens to hundreds of billions of parameters. \\
    These advancements have led to a continual enhancement in their language understanding and generation capabilities, facilitating their \\
    widespread adoption across domains such as natural language processing, intelligent customer service, and automated writing. \\
    Concurrently, trends in transfer learning and open-source sharing have further propelled their applicability and adaptability \\
    across specific tasks and domains. However, as the parameter count escalates, the demand for computational and storage resources also increases, \\
    posing a series of challenges and bottlenecks. This survey aims to analyze the existing issues and relevant research efforts in memory management, process scheduling, and computational resource allocation, offering comprehensive insights and valuable perspectives derived from existing literature.
\end{abstract}

\begin{IEEEkeywords}
    LLM, Transformer, Memory Management, Attention Mechanism, Process Scheduling, Distributed Training, Offloading, Inference
\end{IEEEkeywords}

\section{Introduction}
During the operation of LLMs, the memory requirements are complex and substantial. In the model training phase, memory is needed for model parameters and gradients, activation values, and optimizer states. During the inference phase, memory is required for model parameters and input data buffers. Recently, as the demand for long-sequence inference has grown, the proportion of memory occupied by the Key-Value cache has increased in order to maintain more contextual information, becoming a focal point for memory optimization.
Therefore, efficient memory allocation is crucial for large language models. Memory not only affects data storage but also plays a crucial role in improving the training and inference speed of the model, reducing memory footprint and resource allocation, and preventing memory leaks.

The Transformer model has achieved tremendous success in the field of Natural Language Processing (NLP) and has become the foundational architecture for many powerful pre-trained language models, such as BERT and GPT. LLMs typically rely on the Transformer architecture and leverage massive corpora for pre-training to learn rich representations of language. Thus, LLMs can be viewed as specific applications of the Transformer model in NLP tasks. However, advanced models may have large volumes and require significant resources, resulting in high usage costs. For instance, GPT-3 has 175 billion parameters, and serving models of such scale may incur substantial costs due to computational expenses, necessitating acceleration from the perspective of process management and scheduling.
The inference of LLMs is constrained by memory input/output (I/O) limitations rather than computational constraints. In other words, currently, the time required to load 1MB of data into the computing cores of a GPU is greater than the time these computing cores need to perform LLM computations on 1MB of data. This implies that the throughput of LLM inference is largely dependent on how much batch processing can be accommodated within the high-bandwidth GPU memory.
\section{memory management}
\subsection{Memory Optimization Based on the Attention Algorithm}
\subsubsection{PagedAttention}
In long-sequence inference, the KV cache records a large amount of contextual information related to keys and values. Its memory consumption is substantial and dynamically increases or decreases, similar to the issues encountered in operating systems when allocating memory to processes. In classic operating system memory management methods, virtual memory and paging techniques can effectively reduce unnecessary memory usage, thereby improving memory utilization. The same principles can be applied to the design of LLMs.

In the Transformer architecture and its variants, the attention mechanism is a key component that helps the model understand the relationships between input data. Based on this, incorporating paging concepts such as paging the KV cache should be considered within the Attention algorithm. A typical implementation method is the PagedAttention algorithm proposed by Woosuk Kwon and others (Efficient Memory Management for Large Language Model Serving with PagedAttention). Based on this, a service system for LLMs, known as vLLM, has been developed.

vLLM is based on the PagedAttention algorithm. The strategy of PagedAttention involves dividing the KV cache of each sequence into multiple blocks, each containing a certain number of tokens' keys and values. For each request to the KV cache, multiple non-contiguous blocks can be allocated. (figure)

After dividing the KV cache, the original k and v values in the self-attention calculation formula are transformed into k and v vectors corresponding to the blocks in which they are stored. The formula is modified as follows:

vLLM employs block-level memory management and preemptive request scheduling, both of which are co-designed with PagedAttention. A request’s KV cache is represented as a series of logical KV blocks, filled from left to right as new tokens and their KV cache are generated. Just like the page table in operating system, the KV block manager maintains block tables—the mapping between logical and physical KV blocks of each request. Each block table entry records the corresponding physical blocks of a logical block and the number of filled positions. Thus, with logical block number, physical block number and block table, vLLM can dynamically allocate the KV cache memory to certain request. Moreover, the vLLM’s block table has a “filled” record for each block, which represents the number of tokens in that block. If the “filled” record does not equal the maximum capacity that each block can accommodate, then when the key and value of a new token need to be added to the KV cache, the system will first attempt to fill any vacancies in the previous block. If there are no vacancies, a new physical block will be allocated to the next logical block.

The concept of partitioning the KV cache, inspired by paging and virtual memory in operating systems, allows for similar benefits in optimizing KV cache storage space: by allocating storage in blocks, the space can be non-contiguous, eliminating the external fragmentation caused by contiguous storage. Furthermore, when a request starts or finishes generation, KV blocks can be dynamically allocated and reclaimed, facilitating dynamic memory allocation.

Additionally, dividing the KV cache into blocks for computation enables the PagedAttention kernel to process the KV cache in parallel across multiple positions, thereby enhancing hardware utilization and reducing latency.

\subsubsection{Analysis: Combination of vLLM and distributed storage}
In vLLM, using a centralized scheduler to coordinate parallel model execution is crucial. This strategy ensures that each model shard can efficiently handle the tasks assigned to it. In this setup, although each GPU processes a part of the entire input, all GPUs still need to access the same set of KV cache data. Therefore, vLLM adopts a centrally managed single KV cache manager to synchronize and share critical data across all computing nodes.

vLLM's design is highly suitable for integration with distributed storage systems, offering the some key advantages: vLLM can implement block allocation and dynamic management of the KV cache. When integrated with distributed storage systems, this architecture can leverage the storage system's dynamic resource allocation capabilities, automatically expanding or contracting storage resources as needed. This not only improves memory utilization but also increases the system's total storage capacity. It also helps to store data close to the computing resources that use this data, thereby reducing data transmission times and enhancing overall processing speed. Moreover, this combination can utilize advanced data backup and fault tolerance mechanisms. In the event of a single node or device failure, the system can automatically recover KV cache data from backups.



\subsubsection{Tree Attention}
In traditional long-term language models (LLMs), the keys (k) and values (v) for each output sequence are calculated independently and stored in a kv cache. This method does not effectively utilize the potential data sharing among sequences with the same prefixes. Given that LLMs process tokens sequentially, we can envision each token as a node in a tree, with all potential subsequent tokens as its children. This tree-like structure would facilitate the sharing and reuse of previously computed k and v values during generation, thus enhancing efficiency and reducing memory consumption.

A typical example of the application of this idea is the SpecInfer "Speculative Inference" engine.(SpecInfer: Accelerating Large Language Model Serving with Tree-based Speculative Inference and Verification by Xupeng Miao et al. ) It utilizes a "Small Speculative Model" (SSM), which is much less computationally expensive than an LLM, to perform speculative inference in a tentative manner, carrying out multi-step reasoning. The inference results of the SSM are aggregated into a Speculated Token Tree, which is then verified by the LLM. Efficient tree decoding operators enable parallelized inference, and the paths that are verified are output as the inference result sequence of the model. During this process, sequences with the same prefix can utilize their common ancestors in the tree to share k,v values.

Before performing Tree Attention, SpecInfer decomposes the sentence into multiple tokens and utilizes the SSM to construct candidate token sequences layer by layer, starting from the root node.

During the validation process of Tree Attention, the accuracy of each candidate token sequence must be verified. Tree Attention calculates the probability scores for sequences formed by appending each possible subsequent token to the current prefix. Using this method, SpecInfer achieves parallel verification of multiple tokens. Starting from the tree's root node, each layer's computation can reuse the k and v values calculated in the previous layers. As a result, the KV cache does not store duplicate k and v values for the same tokens, significantly reducing memory consumption.

DFS (Depth-First Search) is an algorithm for traversing tree structures. It starts at the root node and explores along a branch down to a leaf, then backtracks to an earlier node to continue exploring other branches. In SpecInfer, DFS is employed to verify candidate token sequences within a token tree. Specifically, starting from the root, DFS systematically traverses the entire token tree layer by layer, verifying whether each node's candidate token sequence is consistent with the output of LLM. By utilizing DFS, SpecInfer efficiently verifies the entire token tree, ensuring the high quality of the generated results.

By integrating Tree Attention with DFS, redundant KV cache allocations caused by multiple output sequences sharing the same prefixes can be effectively eliminated, significantly improving memory utilization. Moreover, this approach enables parallel processing of tokens, thereby enhancing computational efficiency.
\subsubsection{Token Attention}
In Long-Term Language Models (LLMs), memory resources are allocated for the computation of each token. However, not all tokens are actively used at all times. The memory management for each token in LLMs is not sufficiently detailed, leading to unnecessary occupation of memory resources, which can affect the overall performance and efficiency of the system.

Based on this concept, the LightLLM model was designed(), which includes Token Attention that reduces memory usage by enabling more precise operations on each token.

In traditional self-attention mechanisms, the queries, keys, and values for all tokens are computed simultaneously for the entire sequence. In contrast, in Token Attention, each token independently calculates its own query, key, and value, allowing the model to handle the relationships and interactions between different tokens more flexibly.

The fine-grained management of Token Attention, compared to traditional Attention algorithms, is easier to implement in practice. It can effectively manage memory usage and reduce issues with memory fragmentation.

\subsubsection{Summary}
When employing memory optimization strategies based on attention mechanisms, the reduction in memory usage is often not significant, but there is usually a substantial increase in the inference time for LLMs. Essentially, this strategy involves trading time for space. Therefore, when implementing such optimizations, it is crucial to balance computational performance and memory efficiency, making decisions based on specific application requirements.

\subsection{Memory Management in Long-Term Memory and Recall}
In the description above, the term "memory" has two different meanings. The first "memory" refers to the computer's hardware resources, while the second "memory" refers to the recollection of key contextual information that LLMs need to retain when processing lengthy texts.
\subsubsection{Background}
With the development of large language models (LLMs), the datasets they rely on have become increasingly vast, and the input they process has become more verbose. When faced with massive amounts of input data, the model's original context-linking capabilities might experience a "forgetting" phenomenon, which is the loss of crucial historical information. To address this issue, memory modules were developed. These modules have become effective tools for handling long-term dependencies, enhancing understanding of context, and improving memory of past information.

There are several main types of memory modules in LLMs, including Long Short-Term Memory networks (LSTM), Neural Turing Machines (NTM), Differentiable Neural Computers (DNC), and others. The following will provide a detailed analysis of a specific memory module design called “Self-Controlled Memory System”(SCM).

Enhancing Large Language Model with Self-Controlled Memory Framework(Bing Wang)

\subsubsection{Analysis of SCM}
The SCM system, comprising the language model agent, memory stream, and memory controller, optimizes LLM processing and responsiveness. The memory stream employs caching (e.g., Redis) and vector databases (e.g., Pinecone) for accelerated data access, facilitating structured storage of memory items such as interaction indices, observations, system responses, summaries, and embeddings for efficient data retrieval. It categorizes memories into Activation Memory for historical data and Flash Memory for recent interactions, enabling flexible memory utilization.

In the Memory Controller, Memory Activation relies on prompts to determine the activation of historical memories based on the model's response to current observations. Memories are retrieved using observations as queries, ranked by recency and relevance scores computed from cosine similarity between query and memory embeddings. The top k memories (3 to 10) are activated, ensuring only pertinent and recent memories are utilized.

Summaries are employed to address user queries when activated memory tokens exceed 2000 and memory length exceeds 800 tokens. If a summary suffices to answer the query, it represents the memory, reducing data processing volume and ensuring the model remains within its maximum input length capacity.

Given the extensive range of memory module designs in LLMs, I have not yet fully explored all these designs. Based on my current knowledge and preliminary understanding of this field, here are some of my thoughts and ideas on memory optimization.
Eviction Policies:
Implementing eviction mechanisms to remove memories that have not been used for an extended period can effectively reduce space consumption. This strategy is analogous to cache eviction algorithms in computer systems and allows for dynamic memory management, ensuring operational efficiency.
Distributed Memory Modules:
Applying memory modules in a distributed system to enable shared memory across multiple nodes can enhance the effects of parallel processing when handling large-scale data. This shared approach alleviates the load on individual nodes and accelerates data processing, particularly useful in environments with high concurrency demands.
Paging Mechanism:
Employing a paging mechanism to manage the storage of memory contents allows for dynamic memory allocation and the use of virtual memory. This method increases the utilization and flexibility of memory resources, enabling the efficient storage and management of extensive memory data even under physical memory constraints.
Memory Pooling:
Manage memory allocations using memory pools to reduce the overhead associated with frequent memory allocation and deallocation. Memory pools pre-allocate a large block of memory and dynamically allocate or reclaim memory from this block as needed.
Garbage Collection Optimization:
Optimize garbage collection (GC) mechanisms to minimize their impact on system performance. By improving GC algorithms, such as incremental GC, parallel GC, or real-time GC, memory can be managed more effectively without sacrificing response times.
Heterogeneous Storage Strategies:
Employ a combination of different types of storage media (e.g., RAM, SSD, HDD) to store data with varying access frequencies. Frequently accessed "hot" data can be stored on faster media like RAM, while "cold" data can be placed on lower-cost media like HDD.



\section{process scheduling}

Inference serving plays a pivotal role in the functionality of interactive AI applications based on LLMs, demanding low job completion times (JCT) to ensure an engaging user experience. However, the substantial size and complexity of LLMs present challenges to inference serving infrastructure, often requiring enterprises to deploy costly clusters equipped with accelerators like GPUs and TPUs. LLM inference possesses unique characteristics that set it apart from other deep neural network (DNN) model inference tasks, such as ResNet【6】. While DNN inference jobs are typically deterministic and highly predictable 【7】, LLM inference follows a distinct autoregressive pattern. Each iteration of an LLM inference job generates one output token, which is then appended to the input for producing the next output token in the subsequent iteration. The execution time of LLM inference depends on both the input and output lengths, with the latter being unknown beforehand, and that also gives a direction of optimizing.
Therefore, we start from the Transformer model and analyze various process management and scheduling methods available.


\subsection{iteration-level scheduling in Orca}
\subsubsection{problems to be solved}


Due to the large scale of models and texts, batch processing is employed to improve efficiency. Traditional LLM serving systems adopt a run-to-completion approach, where a task continues execution until completion without interruption or pause. However, this approach presents issues:
- The input lengths vary, and different requests may require different numbers of iterations, causing some requests to complete prematurely.
Assume there are two requests named x1 and x2. x1 carries a string of “I think this is great”. x2 carries a string of “I love you”. We send the two requests to the transformer
\begin{figure}[htbp]
    \centerline{\includegraphics[width=0.5\textwidth]{process fig1.png}}
    \caption{scheduling execution at the request granularity level}
    \label{fig}
\end{figure}
The figure illustrates the scheduling execution at the request granularity level. In this scenario,
although request x2 finishes earlier than request x1, the engine performs computation for both “active” and “inactive” requests throughout all iterations. Such extra computation for inactive requests (x2 at iter 3 and 4) limits the efficiency of batched execution. What makes it even worse is that this behavior prevents an early return of the finished request to the client, imposing a substantial amount of extra latency. This is because the engine only returns the execution results to the serving system when it finishes processing all requests in the batch. Similarly, when a new request arrives in the middle of the current batch’s execution, the newly arrived request must wait until all requests in the current batch have finished. Therefore, this will incur a significant amount of additional computational costs and time overhead. [1]


\subsubsection{Solution: iteration-level scheduling}


Orca gives a distributed serving system which is utilized for deploying Transformer-based generative models.


Orca addresses this by scheduling each request at a finer granularity (i.e., at each iteration), in FCFS order:
- The iteration-level scheduler repeats the following three steps:
- Selects requests to run next.
- Invokes the engine to execute one iteration for the selected requests.
- Receives execution results for the scheduled iteration.

When the scheduler receives returns after each iteration, it can detect the completion status of requests, promptly returning completed requests to the client. Importantly, new incoming requests only need to wait for one iteration, significantly reducing queuing delays.

\subsubsection{Benefits transferability of the iteration-level scheduling strategy}


The transition of the serving system from request granularity to implementing scheduling at each iteration granularity has enhanced overall scheduling flexibility and improved utilization of computational resources.
The scheduler monitors the completion status of requests at each iteration, allowing each completed request to return immediately without waiting for other requests to finish together. The space freed up by returned requests enables new incoming requests to enter, significantly reducing queuing delays.

According to evaluation on a GPT-3 175B model , Orca can significantly outperform NVIDIA FasterTransformer in terms of both latency and throughput: 36.9X throughput improvement at the same level of latency.[1]


The ability of Large Language Models (LLMs) to perform iteration-level process scheduling and its transferability across different LLM architectures stem from their intrinsic model-agnostic nature. Such scheduling methodologies operate independently of the internal structure of the models, focusing instead on task partitioning and management during iterative computations. Leveraging the parallelism inherent in LLM iterative computations, these scheduling approaches efficiently distribute tasks, irrespective of model specifics, thereby facilitating their migration across diverse LLM frameworks. Furthermore, the reliance on task similarities and the adaptability of scheduling algorithms enable seamless transition and optimization across model boundaries, ensuring their applicability in varied academic contexts.

Thus the iteration-level batch processing (also known as continuous batch processing) method is versatile and can be applied to other models as well, and three examples are shown here:
- Utilizing continuous batch processing and model-specific memory optimizations (using vLLM) can increase throughput by up to 23 times. [2]
- Throughput with continuous batch processing (in Ray Serve and Hugging Face's text generation inference) is 8 times that of naive batch processing. [3]
- By employing optimized models (such as NVIDIA's FasterTransformer), throughput can be increased by 4 times compared to naive batch processing. [3]

\subsection{FastServe}
\subsubsection{problems to be solved}

Orca implementations used an FCFS strategy without preemptive mechanisms, meaning once a job enters the running state, it must wait until completion. This leads to severe head-of-line blocking issues,which indicates a phenomenon that the delivery of certain packets or tasks is delayed due to the presence of other packets or tasks ahead of them in the queue. Therefore, FastServe, a distributed inference serving system for LLMs, adopts a skip-join Multi-Level Feedback Queue scheduler to implement preemptive scheduling.
\subsubsection{Skip-join Multi-Level Feedback Queue scheduler}


FastServe is based on a preemptive Multi-Level Feedback Queue, where each job initially enters the highest-level queue and gradually descends to lower priority queues as it consumes its time slice, reducing job completion time. To avoid starving, a promote mechanism periodically promotes some jobs. Specifically, each job sets a starve timer, and if it waits in the waiting state beyond a threshold, it is promoted to the highest-level queue Q1. Finally, the scheduler selects jobs for execution based on queue priority from highest to lowest.

\begin{algorithm}
    \caption{Skip-Join MLFQ Scheduler}
    \begin{algorithmic}[1]
        \State \textbf{Input:} Queues $Q_1, Q_2, ..., Q_k$, jobs $J_{in}$, $J_{pre}$, profiling info $P$
        \State \textbf{Output:} Jobs to be executed $J_{out}$
        \Procedure{SkipJoinMLFQScheduler}{}
        \State \textbf{Initialization:} $J_{out} \leftarrow \emptyset$
        \For{$job \in J_{in}$}
        \State $nextIterTime \leftarrow P.getNextIterationTime(job)$, $p_{job} \leftarrow getHighestPriority(nextIterTime)$
        \State $Q_{p_{job}}.push(job)$
        \EndFor
        \For{$job \in J_{pre}$}
        \State $job.outputNextGeneratedToken()$, $p_{job} \leftarrow job.getCurrentPriority()$
        \If{$job.isFinished()$}
        \State $Q_{p_{job}}.pop(job)$
        \State \textbf{continue}
        \EndIf
        \If{$job.needsDemotion()$}
        \State $nextIterTime' \leftarrow P.getNextIterationTime(job)$
        \State $p_{job} \leftarrow getDemotionPriority(p_{job}, nextIterTime')$, $r.demoteTo(Q_{p_{job}})$
        \EndIf
        \EndFor
        \For{$q \in \{Q_2, Q_3, ..., Q_k\}$}
        \For{$job \in q$}
        \If{$job.needsPromotion()$}
        \State $job.promoteTo(Q_1)$, $job.reserveTime()$
        \EndIf
        \EndFor
        \EndFor
        \For{$q \in \{Q_1, Q_2, ..., Q_k\}$}
        \For{$job \in q$}
        \If{$job.isReady()$ and $|J_{out}| < MaxBatchSize$}
        \State $J_{out}.push(job)$
        \EndIf
        \EndFor
        \EndFor
        \EndProcedure
    \end{algorithmic}
\end{algorithm}
In MLFQ eviction, a skip-join mechanism is introduced. Before a new job joins the queue, it is evaluated (the reason is explained in next paragraph), and based on the estimated overall execution time using prefill time, it is placed into different priority queues. This aims to prioritize the execution of projects with the least remaining time.

\begin{figure}[htbp]
    \centerline{\includegraphics[width=0.5\textwidth]{process fig4.png}}
    \caption{Skip-join MLFQ with starvation prevention}
    \label{fig}
\end{figure}

Specifically, although the number of iterations (i.e., the output length) is not known ahead of time, the execution time of each iteration is predictable. The iteration time is determined by a few key parameters suchas the hardware, the model, and the input length, and thus
can be accurately profiled in advance. Figure below shows the iteration time for GPT-3 2.7B on NVIDIA A100 under different input sequence length. We can see that the first iteration
time (i.e., the execution time to generate the first output token) is longer than those in the decoding phase within a single job. As the input sequence length increases, the first iteration time grows roughly in a linear manner, while the increase of the iteration time in the decoding phase is negligible. This gives the skip-join mechanism information, so the algorithm can decide which queue the new job joins.

\begin{figure}[htbp]
    \centerline{\includegraphics[width=0.5\textwidth]{process fig5.png}}
    \caption{ The execution time of the first four iterations (i.e.first four output tokens) with different input sequence length}
    \label{fig}
\end{figure}

\subsection*{Benefits and drawbacks of the strategy}

\section{computation resource management}
\subsection{distributed training}
With the advancement of LLMs, both model parameters and training datasets are rapidly becoming enormous. To train a mature large-scale model effectively, it is inevitable to employ multiple GPUs for distributed collaborative training to tackle the massive datasets and model parameters that cannot be accommodated on a single card. The crucial issue for large models is how to utilize multi-GPU resources and coordinate information among multiple GPUs. Distributed training nowadays primarily focuses on parallel training through the following three aspects:

\subsubsection{Data parallelism}
Data parallelism involves partitioning massive datasets and maintaining complete models on each GPU. The dataset is divided into 𝑛 parts, and each GPU trains a portion to achieve the training effect. Here, we can discuss two methods of data parallelism in PyTorch classes: DP and DDP.

To achieve effective training across multiple GPUs, it implies that there must be a certain level of information exchange and synchronization among the GPUs. In the torch.nn.DataParallel method, each GPU communicates through a server GPU, responsible for interacting with each GPU, handling information exchange, and summarizing data. Specifically, the server GPU needs to aggregate the forward outputs from each GPU to calculate the loss, compute the gradient of the final layer through the loss, distribute this gradient to each GPU, where each GPU conducts backward propagation independently, and finally aggregate the gradients of all parameters for gradient updates.

Due to the existence of a central GPU, the process is relatively straightforward, with this GPU continuously handling information exchange.
However, the major drawback of this method lies in the central GPU, as its memory and computational capacity become bottlenecks, resulting in low utilization of other GPUs. Hence, this method is not recommended.

With DDP, the most significant structural change is the elimination of the central GPU, thus achieving true distributed effects. To achieve this, we must first introduce a communication method: ring All-Reduce.

Allreduce: a sum or average of partitioned data which is then shared back to the cluster nodes.(https://arxiv.org/pdf/1312.3020)In the framework of ring All-Reduce, each GPU holds an equal status, where each GPU only receives data from its left neighbor and sends data to its right neighbor. The entire ring All-Reduce algorithm consists of two steps:
1) scatter-reduce. This step gradually exchanges and merges gradients with each other, resulting in each GPU holding a part of the fully merged gradient. 2) all-gather. This step involves the gradual exchange of incomplete merged gradients, eventually ensuring that all GPUs possess the complete merged gradient.
The entire architecture forms a ring, where information flows cyclically. This enables the successful interaction and merging of all information, hence the term "ring All-Reduce". The theoretical data transmission volume is 2(N−1)×K/N​, where N is the number of GPUs and K is the total data volume per device. The data transferred approaches 2K, while K×N represents the total data volume. Thus, with an increasing number of GPUs, ring All-Reduce can achieve linear scaling.

Returning to DDP, with the premise that ring All-Reduce can achieve linear scaling, we abandon the central GPU and opt for a more evenly balanced distributed system. This is precisely what DDP accomplishes. Therefore, the gradient aggregation function performed by the central GPU in DP is replaced in DDP by allreduce among GPUs. Moreover, since backward does not depend on the aggregated gradient, the allreduce process can be synchronized with backward, further enhancing concurrency and utilization of computational resources.

\subsubsection{Pipeline Parallelism}
Data parallelism only addresses the issue of oversized datasets. In today's LLMs, many model parameters reach several billion, and ChatGPT-4 even surpasses a trillion parameters. In such cases, for distributed training to occur, it is necessary to partition the model.
natural approach is to split the model between different layers, with each GPU holding different layers. Data flows layer by layer between GPUs, sequentially forwarding and backwarding through each layer. This approach solves the problem of storing excessively large models. However, merely doing so doesn't improve computation speed with multiple cards due to communication overhead. In fact, it may prolong training time since only one GPU computes at a time while others remain idle, failing to utilize computational resources effectively.
Pipeline parallelism, inspired by CPU instruction pipelining, introduces GPU computation pipelining in distributed training. The dataset is further divided, with each GPU handling a micro-batch. Once processed, the output is sent to the next GPU, while the current GPU proceeds to handle the next micro-batch. This pipeline design ensures that GPUs are continuously engaged, fully utilizing all computational resources. Consequently, linear acceleration is achievable with an increasing number of GPUs.

\subsubsection{Tensor Parallelism}
As models grow larger, a single GPU may not even accommodate a single layer. In such cases, we consider vertically dividing layers into segments and distributing them across different GPUs. This is where tensor parallelism comes into play.
A tensor, essentially an n-dimensional matrix, is broken down since computations in frameworks like Transformers mainly involve matrix multiplications with linear relationships between parameters. Thus, the overall matrix is divided and matrix multiplications are performed separately, with results subsequently aggregated.
Tensor parallelism entails dividing internal matrices of large models, conducting separate computations, and then aggregating or concatenating outputs using techniques like allreduce. This method is prevalent in LLMs, primarily because modern LLMs are predominantly built upon Transformer architectures, which facilitate tensor parallelism. In contrast, for CNNs, the convolution process is more complex, making it challenging to parallelize through simple matrix partitioning. Hence, this stark difference in parameter quantity between LLMs and CNNs is a significant reason for the former's much larger parameter count.

\subsubsection{ZeRO(Zero Redundancy Optimizer)}
After partitioning and parallelizing datasets and models, as the number of parameters increases, it's evident that certain redundant parts occupy a significant amount of memory. These parts can be further optimized, with one of them being model states.
Model states primarily consist of three components: 1) optimizer states, 2) gradient, 3) model parameter.
ZeRO technology emerges to address the issue of model states consuming a large amount of memory. ZeRO is divided into three levels: ZeRO-1 handles optimizer states, ZeRO-2 handles optimizer states and gradients, and ZeRO-3 handles all model states.
The concept behind ZeRO is relatively straightforward: distribute model states across different GPUs, and during specific steps where complete model states are required, obtain them through allreduce. This increases communication overhead but reduces memory consumption. However, as the number of partitions increases, the corresponding technical intricacies also escalate, and excessive communication may lead to diminishing returns.

\subsubsection{3D Hybrid Parallelism}
By integrating data parallelism, tensor parallelism, pipeline parallelism, and ZeRO technology, with ZeRO-3 inherently partitioning model parameters, overlapping functionalities with tensor parallelism and pipeline parallelism occur. Meanwhile, ZeRO-2 partitions gradients. Combining ZeRO-2 with pipeline parallelism results in significant communication overhead. Hence, when employing 3D parallelism, it is advisable to combine it with ZeRO-1.
Through the coordinated use of multiple parallel strategies, large model parameters can reach the scale of hundreds of billions.

\subsection{Offloading Inference}
For large-scale model training networks, inference is undoubtedly much easier compared to training. However, current commercial inference for large models, such as ChatGPT, is conducted on remote servers, with results transmitted back to users over the network. If it were possible to deploy pre-trained large models locally for inference, it would undoubtedly further expand the application of large models, representing a significant step forward.

When it comes to LLM inference, there are three main directions for reducing computation and storage resources:
1) model compression 2) collaborative inference 3) offloading.
Offloading is one aspect to reduce computation and storage resources for LLM inference. it makes full use of various hardware resources, transferring parameters from GPU memory to system memory or even disk storage.
Let's delve into two models specifically designed for offloading:

\subsubsection{Flexgen}
FlexGen[1] aims for the utmost throughput, identifying the most suitable offloading strategy within a specified search space. It utilizes methods like model compression and sparsification to execute inference for models with hundreds of billions of parameters on consumer-grade hosts.
FlexGen can be flexibly configured under various hardware resource constraints by aggregating memory and computation from the GPU, CPU, and disk. Through a linear programming optimizer, it searches for the best pattern to store and access the tensors, including weights, activations, and attention key/value (KV) cache. FlexGen further compresses both weights and KV cache to 4 bits with negligible accuracy loss.

One key idea of FlexGen is to play the latency-throughput trade-off. Achieving low latency is inherently challenging for offloading methods, but the I/O efficiency of offloading can be greatly boosted for throughput-oriented scenarios (see the figure above). FlexGen utilizes a block schedule to reuse weight and overlap I/O with computation, as shown in figure (b) below, while other baseline systems use an inefficient row-by-row schedule, as shown in figure (a) below.
\begin{figure}[htbp]
    \centerline{\includegraphics[width=0.5\textwidth]{device fig2.jpg}}
    \caption{two defferent schedules. The red arrows donate the computation order}
    \label{fig}
\end{figure}

\subsubsection{PowerInfer}
PowerInfer[2] partitions neurons into hot and cold neurons based on predictors. During inference, hot neurons are computed on the GPU, while cold neurons are processed on the CPU.
PowerInfer's core idea is essentially sparsity, built upon the fact that only a small portion of neurons are active during the neural network inference process, while the majority remain inactive. Dejavu experiments have verified that over 80\% of neurons and over 95\% of MLP parameters can be excluded from inference. This implies that a subset of neurons can produce results similar to those of the entire large model. Therefore, placing cold neurons on the CPU for computation would not significantly affect the inference results, as long as the active neurons are computed effectively. Additionally, since cold and hot neurons are pre-determined, the data exchange and communication overhead during inference would be considerably reduced.
\begin{figure}[htbp]
    \centerline{\includegraphics[width=0.5\textwidth]{device fig1.png}}
    \caption{The architecture overview and inference workflow of PowerInfer.}
    \label{fig}
\end{figure}
The final implementation of PowerInfer has shown remarkable effectiveness. When compared with llama.cpp across various models such as Opt 66B, Opt 175B, Falcon 40B, and Llama 70B, it has achieved several-fold acceleration for different output lengths and data precisions.

PowerInfer's approach is novel, particularly in its prediction of cold and hot neurons, which offers an innovative perspective. The demonstration results are quite impressive, showcasing significant speed improvements. However, the practicality of this approach still needs improvement. Its model adaptability is somewhat cumbersome, as it is only suitable for models using the ReLU activation function. Models utilizing SILU activation require fine-tuning for a ReLU version. Additionally, this method is lossy, which raises questions about its compatibility with further developments in quantization. Whether PowerInfer can coexist with quantization is also a worthwhile consideration.

\section*{Acknowledgment}


The preferred spelling of the word ``acknowledgment'' in America is without
an ``e'' after the ``g''. Avoid the stilted expression ``one of us (R. B.
G.) thanks $\ldots$''. Instead, try ``R. B. G. thanks$\ldots$''. Put sponsor
acknowledgments in the unnumbered footnote on the first page.

\section*{References}

Please number citations consecutively within brackets \cite{b1}. The
sentence punctuation follows the bracket \cite{b2}. Refer simply to the reference
number, as in \cite{b3}---do not use ``Ref. \cite{b3}'' or ``reference \cite{b3}'' except at
the beginning of a sentence: ``Reference \cite{b3} was the first $\ldots$''

Number footnotes separately in superscripts. Place the actual footnote at
the bottom of the column in which it was cited. Do not put footnotes in the
abstract or reference list. Use letters for table footnotes.

Unless there are six authors or more give all authors' names; do not use
``et al.''. Papers that have not been published, even if they have been
submitted for publication, should be cited as ``unpublished'' \cite{b4}. Papers
that have been accepted for publication should be cited as ``in press'' \cite{b5}.
Capitalize only the first word in a paper title, except for proper nouns and
element symbols.

For papers published in translation journals, please give the English
citation first, followed by the original foreign-language citation \cite{b6}.

\begin{thebibliography}{00}
    \bibitem{b1} G. Eason, B. Noble, and I. N. Sneddon, ``On certain integrals of Lipschitz-Hankel type involving products of Bessel functions,'' Phil. Trans. Roy. Soc. London, vol. A247, pp. 529--551, April 1955.
    \bibitem{b2} J. Clerk Maxwell, A Treatise on Electricity and Magnetism, 3rd ed., vol. 2. Oxford: Clarendon, 1892, pp.68--73.
    \bibitem{b3} I. S. Jacobs and C. P. Bean, ``Fine particles, thin films and exchange anisotropy,'' in Magnetism, vol. III, G. T. Rado and H. Suhl, Eds. New York: Academic, 1963, pp. 271--350.
    \bibitem{b4} K. Elissa, ``Title of paper if known,'' unpublished.
    \bibitem{b5} R. Nicole, ``Title of paper with only first word capitalized,'' J. Name Stand. Abbrev., in press.
    \bibitem{b6} Y. Yorozu, M. Hirano, K. Oka, and Y. Tagawa, ``Electron spectroscopy studies on magneto-optical media and plastic substrate interface,'' IEEE Transl. J. Magn. Japan, vol. 2, pp. 740--741, August 1987 [Digests 9th Annual Conf. Magnetics Japan, p. 301, 1982].
    \bibitem{b7} M. Young, The Technical Writer's Handbook. Mill Valley, CA: University Science, 1989.
\end{thebibliography}
\vspace{12pt}
\color{red}
IEEE conference templates contain guidance text for composing and formatting conference papers. Please ensure that all template text is removed from your conference paper prior to submission to the conference. Failure to remove the template text from your paper may result in your paper not being published.

\end{document}
